\documentclass[a4paper,12pt]{article}
\usepackage[utf8]{inputenc}
\usepackage[russian]{babel}
\usepackage{hyperref}
\usepackage{graphicx}
\usepackage{listings}
\usepackage{color}

\definecolor{keyword}{rgb}{0.36, 0.54, 0.66}
\definecolor{comment}{rgb}{0.0, 0.5, 0.0}
\definecolor{string}{rgb}{0.58, 0.0, 0.82}

\lstset{
  language=C++,
  keywordstyle=\color{keyword}\bfseries,
  commentstyle=\color{comment}\itshape,
  stringstyle=\color{string},
  basicstyle=\ttfamily\small,
  breaklines=true,
  showstringspaces=false
}

\title{Документация проекта Snake}
\author{fearowpi}
\date{\today}

\begin{document}

\maketitle

\tableofcontents
\newpage

\section{Описание проекта}
Проект реализует игру Snake/Tetris с использованием библиотеки \texttt{ncurses}.  
Игра поддерживает:
\begin{itemize}
    \item управление с клавиатуры,
    \item отслеживание счета и рекордов,
    \item динамическое размещение яблок,
    \item уровни сложности,
    \item сохранение рекордов в файл.
\end{itemize}

\section{Файловая структура}
\begin{itemize}
    \item \texttt{src/brick_game/snake} — исходные файлы на C++:
    \begin{itemize}
        \item \texttt{game.h} — основные функции управления игрой и состояние.
        \item \texttt{classes\_snake\_apple.h} — классы \texttt{Snake} и \texttt{Apple}.
        \item \texttt{snake.cpp}
        \item \texttt{apple.cpp}
        \item \texttt{functions\_of\_fields.cpp}
    \end{itemize}
    \item \texttt{Makefile} — сборка проекта.
    \item \texttt{FSM\_snake.png} — диаграмма конечного автомата состояний.
    \item \texttt{highscore\_snake.txt} — файл для хранения рекордов.
    \item \texttt{documentation\_snake.tex} — документация проекта.
\end{itemize}

\section{Основные классы}

\subsection{Класс \texttt{Cell}}
Структура для хранения координат клетки на поле.
\begin{lstlisting}
struct Cell {
  int x;
  int y;
};
\end{lstlisting}

\subsection{Класс \texttt{Snake}}
\begin{itemize}
    \item \textbf{Поля:} 
    \begin{itemize}
        \item \texttt{std::vector<Cell> body} — тело змеи.
        \item \texttt{size\_t length\_of\_snake} — длина змеи.
    \end{itemize}
    \item \textbf{Методы:} 
    \begin{itemize}
        \item \texttt{get\_length\_of\_snake()} — возвращает длину.
        \item \texttt{increase\_length\_of\_snake()} — увеличивает длину.
        \item \texttt{advanceHead()} — перемещает голову змеи.
        \item \texttt{updateTail()} — обновляет хвост змеи.
    \end{itemize}
\end{itemize}

\subsection{Класс \texttt{Apple}}
\begin{itemize}
    \item \textbf{Поля:} координаты яблока \texttt{x, y}.
    \item \textbf{Методы:} 
    \begin{itemize}
        \item \texttt{getX(), getY()} — получить координаты.
        \item \texttt{relocate(Snake \&my\_snake)} — случайное перемещение яблока.
    \end{itemize}
\end{itemize}

\section{Основные функции}

\subsection{Инициализация}
\begin{lstlisting}
void initialization();
\end{lstlisting}
- Сбрасывает поле, создает начальную позицию змеи и яблока, считывает рекорд.

\subsection{Обработка ввода пользователя}
\begin{lstlisting}
void userInput(UserAction_t action);
\end{lstlisting}
- Управляет переходами между состояниями:
\texttt{State\_start, State\_move, State\_pause, State\_terminate, State\_exit}.

\subsection{Движение змеи: функция \texttt{processSnakeMove}}


Главная функция обработки движения змейки.  
Принимает действие пользователя (\texttt{Up}, \texttt{Down}, \texttt{Left}, \texttt{Right}) и выполняет логику шага игры.

\paragraph{Алгоритм работы:}
\begin{enumerate}
  \item Получает текущее состояние змейки и яблока.
  \item Определяет направление движения по первым двум сегментам тела змеи.
  \item Запрещает движение в противоположную сторону (например, если змейка движется вверх, нельзя сразу вниз).
  \item Вычисляет новую позицию головы с помощью функции \texttt{getNextHeadPosition}.
  \item Проверяет условия:
  \begin{itemize}
    \item \texttt{isColliding} --- не столкнулась ли змейка со стеной или своим телом;
    \item \texttt{checkAppleCollision} --- не съела ли змейка яблоко.
  \end{itemize}
  \item Если коллизии нет:
  \begin{enumerate}
    \item Перемещает голову (\texttt{advanceHead}).
    \item Если яблоко не съедено --- убирает хвост (\texttt{updateTail}).
    \item Если яблоко съедено:
    \begin{itemize}
      \item увеличивает длину (\texttt{increase\_length\_of\_snake});
      \item обновляет счёт и уровень;
      \item пересчитывает скорость;
      \item при необходимости обновляет рекорд (\texttt{play\_w\_file}).
    \end{itemize}
  \end{enumerate}
  \item Обновляет игровое поле (\texttt{resetField}, \texttt{fillField}).
\end{enumerate}


\subsection{Работа с полем}
\begin{lstlisting}
void resetField();
void fillField(Snake &snake, const Apple &app);
void resetDynamicField(int a);
\end{lstlisting}
- Сбрасывает или заполняет игровое поле.
- \texttt{resetDynamicField(0)} — выделяет память.
- \texttt{resetDynamicField(1)} — освобождает память.

\section{Сборка проекта}



\begin{itemize}
  \item \texttt{all} --- сборка проекта (библиотека + бинарник).
  \item \texttt{install} --- компиляция и установка игры.
  \item \texttt{uninstall} --- удаление установленного бинарника.
  \item \texttt{clean} --- очистка временных и объектных файлов.
  \item \texttt{test} --- сборка и запуск модульных тестов.
  \item \texttt{dvi} --- генерация PDF-документации из \LaTeX.
  \item \texttt{dist} --- упаковка проекта в архив \texttt{.tar.gz}.
\end{itemize}


\section{Диаграмма конечного автомата}
\begin{center}
\includegraphics[width=0.8\textwidth]{FSM_snake.png}
\end{center}

\section{Файл рекордов}
\texttt{highscore\_snake.txt} используется для хранения максимального счета.  
- Чтение: при старте игры.
- Запись: при достижении нового рекорда.

\section{Используемые технологии}
\begin{itemize}
    \item C++17
    \item ncurses
    \item clang-format
    \item gcov (покрытие тестами)
\end{itemize}

\end{document}
